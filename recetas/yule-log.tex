\subsection{Yule log}

\underline{Ingredientes}

\textbf{Esponja}

\begin{itemize}
\item $\frac{1}{4}$ de taza de harina
\item 2 cucharadas de maicena
\item 2 cucharadas de cocoa
\item $\frac{1}{8}$ de cucharadita de sal
\item $\frac{1}{2}$ taza de az\'ucar
\item 2 huevos enteros
\item 3 yemas de huevo
\item 2 claras de huevo
\item 1 cucharadita de vainilla
\item $\frac{1}{4}$ de cucharadita de crema tartara
\item Un poco de azucar glass
\end{itemize}

\textbf{Relleno}

\begin{itemize}
\item 1 taza de heavy whipping cream
\item $\frac{1}{2}$ cucharadita de vainilla
\item $\frac{1}{4}$ de taza de azucar
\item 2 cucharadas de cocoa
\end{itemize}

\textbf{Ganache}

\begin{itemize}
\item 6 oz de chocolate oscuro
\item 2 oz de chocolate dulce (mil chocolate)
\item $\frac{3}{4}$ de taza de heavy whipping cream (fría)
\item 25 gr de mantequilla
\end{itemize}

\underline{Instrucciones}

\textbf{Esponja}

\begin{enumerate}
\item Cubrir de mantequilla un bandeja, poner un papel encerado, luego otra capa de mantequilla y enharinar.
\item Batir a alta velocidad los huevos enteros, las yemas, la vainilla y la azúcar menos una cucharada (guardar) por \Sim 3 min hasta que este espesa.
\item Añadir la harinna, la maicena, la sal y la cocoa de forma suave para no dejar escapar el aire.
\item Batir la crema tartara y las claras por unos segundos. Añadir la cucharada de azucar y batir a punto de turron.
\item Añadir este ultimo batido a la mezcla de forma suave.
\item Esparcir de forma uniforme en la bandeja.
\item Hornear a 450 F (230 C) por 6 min. Que este bien cocido pero evitar que se seque y endurezca.
\item Espolvorear la azucar glass sobre la esponja.
\item Volvear sobre una toalla
\item Espolvorear azucar glass sobre el otro lado
\item Enrollar y dejar que se enfrie
\end{enumerate}

\textbf{Ganache}

\begin{enumerate}
\item Calentar la whipping cream y la mantequilla hasta que comienze a hervir y sacar de la estufa.
\item Añadir el chocolate y mezclar hasta que se derrita y este uniforme.
\item Dejar reposar hasta que se enfríe y este untable. Si se mete el refri para acelerar el procesos menear cada 10 min.
\end{enumerate}

\textbf{Relleno}
\begin{enumerate}
\item Batir todo hasta que este bien firme
\end{enumerate}

\textbf{Final}
\begin{itemize}
\item Desenrollar la esponja
\item Untar el relleno de forma uniforme por dentro
\item Enrollar
\item Cubrir con el ganache
\end{itemize}

