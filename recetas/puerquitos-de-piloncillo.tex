\subsection{Puerquitos de piloncillo}
\underline{Ingredientes}
\begin{itemize}
\item 500 gr de harina blanca ($\sim 3 2/3$ de taza)
\item 250 gr de piloncillo
\item 1 1/2 taza de agua (360 ml)
\item 3 clavos de olor
\item 2 huevos a temperatura ambiente
\item 1/2 taza de manteca vegetal ($\sim$ 75 gr)
\item 1 cucharada de polvo para hornear ($\sim 9$ gr)
\item 1 cucharadita de bicarbonato de sodio ($\sim 6$ gr)
\item 1 cucharadita de levadura instant\'anea ($\sim 3$ gr)
\item 1 pizca de sal molida
\item 1 cucharadita de a\'uzcar blanca
\item 1/4 de cucharadita de canela en polvo
\item 3 cucharadas de agua ($\sim 30$ ml)
\item 1 cucharadita de extracto de vainilla
\item 1 molde en forma de puerquito

\end{itemize}

\underline{Instrucciones}

\begin{enumerate}
\item Hacer un jarabe. Mezclar la taza y media de agua con el piloncillo, el clavito de holor y la canela en polvo y ponerlo a hervier a fuego alto y mezclar por 15 minutos. Dejar a enfriar. 
\item Poner a fermentar la levadura. Disolver la levadura , la cucharadita de azucar y 1 cucharada de harina en 3 cucharadas de agua tibia. Dejar reposar hasta que duplique su tamaño. 
\item Mezclar la harina con el bicarbonato y el polvo para hornear. 
\item Romper los huevos en recipientes separados. Uno es para barnizar y el otro es para la masa. Batir el huevo para barnizar. 
\item Una vez que se enfríe el jarabe colarlo. Deben salir aproximadamente 1 1/4 de taza o 300 ml
\item Mezclar la harina con la manteca vegetal hasta que esté bien integrada. 
\item Agregar el resto de los ingredientes: huevo, jarabe, sal y levadura. Y amasar aproximadamente por 5 minutos. La masa es bastante aguadita, es normal, no agregar mucha más harina. 
\item Poner la masa en un recipiente tapado y dejar reposar en el refri por 10 minutos. 
\item Engrasar y enharinar 2 charolas 
\item Extender la masa en hasta que quede de aproximadamente 2 cm de grosor. Usar harina para que no se pegue. 
\item Cortar cochinitos. 
\item Retirar el exceso de masa y colocar los puerquitos en charolas engrasadas y enharinadas. 
\item Incorporar el exceso de masa en una bolita (con cuidado de no amasar mucho) y volver a extender para cortar más puerquitos. 
\item Barnizar los puerquitos con el huevo batido
\item Hornear a 180 $^\circ$C o $350^\circ$F hasta que estén doraditos. 
\end{enumerate}
