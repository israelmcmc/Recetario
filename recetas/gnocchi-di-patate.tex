\subsection{Gnocchi di patate}

Basado en \href{https://ricette.giallozafferano.it/Gnocchi-di-patate.html}{Gnocchi di patate en Giallo Zafferano}

\textbf{Ingredientes}
\begin{itemize}
\item \SI{300}{gr} de harina
\item \SI{1}{kg} de papa grande (i.e. russet)
\item 1 huevo
\item Una pizca de sal
\end{itemize}

\textbf{Intrucciones}
\begin{enumerate}
\item Poner las papas en agua hirviendo y cocer hasta que un tenedor entre fácilmente (\SIrange{\sim 30}{45}{min}).
\item Pelar las papas mientras aún estan calientes.
\item Pasar por una prensa de papas\footnote{O aplastas las papas y pasar por colador.}. Debería de quedar \SI{900}{gr} de puré.
\item Mientras el puré de papas ún esta caliente, mezclar con la harina, el huevo y la sal. 
\item Amasar hasta que esté homogéneo, y no más. No debería de tardar más de \SI{5}{min}.
\item En una superficie enharinada con sémola\footnote{La sémola es mejor que la harina para que no se pegue a los gnocchis. No confundir con semolina, que es de grano más grueso.}, rodar una porción de masa y hacer una tira alargada de \SI{\sim1.5}{cm} de diámetro (más o menos el grosor de un dedo)
\item Cortar en trozos de \SI{\sim1.5}{cm} de ancho.
\item Rodar por una tablilla estriada\footnote{O un tenedor} enharinada con sémola, usando el pulgar para aplastarlos libgeramente y dejar una muesca del lado de abajo. Se deben de rodar de forma transversal a como se redó la masa al formar la tira. Ver Fig 
\item Poner en agua hirviendo con sal\footnote{Es importante que la razon entre gnocchis y agua sea grande, para que no baje la temperatura del agua. Aún mas importante que para pasta normal.}\footnote{\SI{2}{cucharaditas} de sal por litro de agua} hasta que suban a la superficie (\SI{\sim 1}{min}).
\end{enumerate}


\begin{figure}
\centering
\includegraphics[width=\textwidth]{recetas/gnocchi-di-patate/figuras/rodado-gnocchi.pdf}
\caption{Así se le dan forma a los gnochis. Es importante que el rodaje sea transversal al corte, es decir, que la parte ``pegajosa'' pase por la tablilla,. De lo contrario los borden no quedan redondos y los zurcos no son tan pronunciados.}
\label{fig:rodado-gnocchi.pdf}
\end{figure}

\textbf{Notas}\\
Si no se van a cocer y comer apenas hechos, para guardar los gnocchis hay que congelarlos de forma individual en charolas. Para cocerlos se echan al agua hirviendo directamente sin descongelar. Es importante usar mucha agua y sólo cocer de pocos a la vez. 