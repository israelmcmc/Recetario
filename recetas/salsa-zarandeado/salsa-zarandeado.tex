\subsection{Salsa para zarandeado}
\label{salsa-zarandeado}

Basada en : \href{https://aprendeacocinarfacil.wordpress.com/2012/03/13/pescado-sarandeado-estilo-sinaloa-aya-pinchi/}{Pescado zarandeado estilo Sinaloa - APRENDE A COCINAR Con el Chef Vizcaino} \\

Ver \hyperref[receta:zarandeado]{\textit{zarandeado}}, aunque también le queda bien pescado en general.\\

\underline{Ingredientes}
\begin{itemize}
\item 5 jitomates en cuadritos
\item 1 chile anaheim o california fresco en aros
\item 1 bara de apio
\item 3 chile serrano
\item 1/2 de cebolla
\item 1/4 de tazade de chicharos
\item 1/4 de taza de aceitunas
\item 1/4 de cucharadita de oregano
\item 1 pizca de comino
\item 3 dientes de ajo machacados
\item 30 gr de mantequilla
\item 2 cucharadas de aceite de oliva
\item Sal al gusto (\SI{\sim 2}{cucharaditas})
\item Pimienta al gusto (\num{\sim 1/2} de cucharadita)
\end{itemize}


\underline{Instrucciones}
\begin{enumerate}
\item Guisar la cebolla y el apio en el aceite.
\item Agregar la mantequilla hasta que se derrita.
\item Agregar el resto de los ingredientes y agua la necesaria (poca).
\item Cocinar por \SI{\sim25}{min}.
\end{enumerate}
