\subsection{Queso panela}

Basada en : \href{http://www.sagarpa.gob.mx/desarrolloRural/Documents/fichasaapt/Elaboraci%C3%B3n%20de%20quesos.pdf}{SAGARPA - Elaboración de quesos tipo Panela y Oaxaca} \\

\underline{Ingredientes}
\begin{itemize}
\item 4 litros de leche pasteurizada \footnote{De preferencia pasteurizada a bajas temperaturas y no homogeneizada. Quitar crema si no está homogeneizada. No usar leche ultra-pasteurizada.}
\item $\sfrac{1}{4}$ de pastilla para cuajar \footnote{Puede variar dependiendo de la marca. Usé Walcoren que cuajaba hasta 
\Sim 20litros por pastilla.}
\item 30-40gr de sal
\item 1 cucharadita de cloruro de calcio al 32\% en peso \footnote{Opcional. Mejora el rendimiento}
\end{itemize}


\underline{Instrucciones}
\begin{enumerate}
\item Calentar la leche a \Sim 32 C
\item Disolver la pastilla para cuajar en un poco de agua y revolver en la leche.
\item Dejar reposar hasta que cuaje y se pueda cortar (30-60 min).
\item Hacer varios cortes especiados por un par de centimetros, en ambas direcciones.
\item Calentar muy lentamente hasta \Sim 32 C mientras se menea para separar el suero.
\item Dejar reposar por 15 min.
\item Con la ayuda de un colador remover el suero, pero sin exprimir.
\item Revolver la cuajada con la sal. Usar 3\% del peso del cuajo.
\item Poner en un molde con estame\~na y exprimir suavemente el suero excedente.
\item Dejar reposar durante 15 min, voltear, y dejar reposar por 3 horas.
\end{enumerate}
