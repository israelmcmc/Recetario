\subsection{Queso panela}

Basada en : \href{http://www.sagarpa.gob.mx/desarrolloRural/Documents/fichasaapt/Elaboraci%C3%B3n%20de%20quesos.pdf}{SAGARPA - Elaboración de quesos tipo Panela y Oaxaca} \\

\underline{Ingredientes}
\begin{itemize}
\item 1 gal\'on (3.8L) de leche pasteurizada \footnote{De preferencia pasteurizada a bajas temperaturas y no homogeneizada. Quitar crema si no está homogeneizada. No usar leche ultra-pasteurizada.}
\item $\sfrac{1}{4}$ de pastilla para cuajar \footnote{Puede variar dependiendo de la marca. Usé Walcoren que cuajaba hasta 
\Sim 20litros por pastilla.}
\item 30-40gr de sal
\item Opcional: $\sfrac{1}{2}$ cucharadita de cloruro de calcio al 32\% en peso \footnote{Mejora el rendimiento, especialmente si la leche fue homogeneizada o pasteurizada a altas temperaturas.}
\end{itemize}


\underline{Instrucciones}
\begin{enumerate}
\item Calentar la leche a \Sim 32 \deg C
\item Opcional: Diluir el cloruro de calcio en $\sfrac{1}{4}$ tazas de agua y agregar a la leche.
\item Disolver la pastilla para cuajar en un poco de agua y revolver en la leche.
\item Dejar reposar hasta que cuaje y se pueda cortar (30-60 min). El cuchillo sale limpio cuando esta lista..
\item Hacer varios cortes especiados por un par de centimetros, en ambas direcciones.
\item Dejar reposar por \SI{15}{min}.
\item Calentar muy lentamente hasta \Sim 32 \deg C mientras se menea por \Sim 15 min para separar el suero.
\item Dejar reposar por 15 min.
\item Con la ayuda de un colador remover el suero\footnote{Has queso ricotta (\ref{queso-ricotta}) con el suero.}, pero sin exprimir.
\item Amasar la cuajada con la sal. Usar 3\% del peso del cuajo.
\item Poner en un molde con estame\~na y exprimir suavemente el suero excedente.
\item Dejar reposar durante 15 min, voltear, y dejar reposar por 3 horas.
\end{enumerate}
