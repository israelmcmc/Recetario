\subsection{Pozole verde}

\underline{Ingredientes}
\begin{itemize}
\item $750 \rm{g}$ de nixtamal (maiz)
\item $2 \frac{1}{2}$ ajos
\item $1 \frac{1}{2}$ cebollas
\item $2$ hojas de laurel 
\item $\sim 9$ litros de agua
\item sal de grano
\item $2$ chiles poblanos
\item $2$ chiles jalapeños
\item $1$ chile serrano
\item $250 \rm{g}$ de tomatillo
\item $1 \rm{tz}$ de pepita de calabaza
\item Hierbas verdes (1 ramita de epazote, 6 hojas de espinaca, 1 ramo de cilantro, 1 ramo de perejil y lechuga)
\item $5$ pimientas negras
\item $3$ clavos de olor
\item Manteca de cerdo
\item $1 \rm{cdita}$ de comino
\item $3 \rm{cdas}$ de oregano
\item $1.5 \rm{kg}$ de carne por cada $750 \rm{g}$ de maiz (maciza, cabeza, espinazo, etc.. al gusto de preferencia con hueso para el caldo).
\end{itemize}

\underline{Instrucciones}
\begin{enumerate}
\item Empezando con maiz nixtamalizado, lavar bien y verificar que este bien descabezado el grano. 
\item En agua caliente, se pone a cocer el maiz, dos ajos enteros (mochos para que suelten sabor pero sin pelar), las hojas de laurel y una cebolla grande (mocha o con una cruz para que suelte el sabor).
\item Se parte la carne y se agrega al maiz, junto con un poco de sal.
\item Cocer tomatillos y chiles jalapeños en agua. Por separado, dorar 3 dientes de ajo y media cebolla en manteca. Al final, se reserva la manteca y se licuan la cebolla y el ajo junto con los tomatillos, chiles, pimientas, clavo y dos tazas de agua. 
\item En otra tanda se licuan el resto de las hierbas (epazote, espinaca, cilantro, etc..) junto con las pepitas tostadas $1 \rm{cda}$ de oregano y sal. 
\item Ahora se recuecen las dos salsas en la manteca y la mezcla se deja hervir a fuego lento con un puñito de oregano triturado.
\item Se sacan los ajos, la cebolla y las hojas de laurel del caldo de puerco y se agrega la salsa verde. Se deja a fuego lento y se agrega sal al gusto.
\item Se sirve con chicharron, cebollita picada finamente, rabano, lechuga, tostadas, aguacate y limon.
\end{enumerate}

