\subsection{Zarandeado}

Basada en : \href{https://aprendeacocinarfacil.wordpress.com/2012/03/13/pescado-sarandeado-estilo-sinaloa-aya-pinchi/}{Pescado zarandeado estilo Sinaloa - APRENDE A COCINAR Con el Chef Vizcaino} \\

\underline{Ingredientes}
\begin{itemize}
\item Un pescado de 2-3 kg \underline{con} espinas \footnote{Pargo o Huachinango, Mero o Cabrilla, Lobina, Curvina, Pampano, Lenguado o Rodaballo, Dorado, o Bonito). Puede ser una parte de un pescado más grande, pero no un pescado más chico.}
\item 1/2 taza de mayonesa
\item 2 cucharadas mostaza
\item 1/2 cucharada de apio en polvo
\item 1/2 cucharada de ajo en polvo
\item 1/2 cucharada de cebola en polvo
\item \hyperref[salsa-zarandeado]{Salsa para zarandeado}
\item Sal
\end{itemize}


\underline{Instrucciones}
\begin{enumerate}
\item Abrir el pescado, dejando todo (incluyendo escamas) excepto las vísceras (\href{https://www.youtube.com/watch?v=_OmdFQWyWRc}{video}):
\begin{enumerate}
\item Hacer un corte largo desde la aleta dorsal. El cuchillo pasa por un lado de las espinas traseras y corta las costillas.
\item Con un cuchillo grande y un mazo partir el craneo en dos, a lo largo.
\item Remover las vísceras con las manos (dejar las agallas).
\item Con el cuchillo y el mazo romper cortar la columna vertebral después de la cabeza antes de la cola.
\item Hacer el mismo corte largo por el otro lado del pescado, también pasando por un lado de las espiras traseras y cortando las costillas.
\item El pescado debe de quedar en forma de mariposa, con la vertebra y las espinas traseras por un lado.
\end{enumerate}
\item Salar el pescado al gusto
\item Mezclar la mayonesa, la mostaza, y los polvos. Adobar el pescado.
\item Asar en la zaranda con fuego algo (de preferencia de leña) por el lado de la carne por \SIrange{15}{20}{min}, hasta que este dorado.
\item Abrir la zaranda y cubrir con la salsa.
\item Poner otra vez al fuego por el otro lado por \SI{\sim 20}{min} más. Las escamas deben de quedar quemadas.
\end{enumerate}
