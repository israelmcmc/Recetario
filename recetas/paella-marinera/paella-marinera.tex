\subsection{Paella marinera}

Basada en: \href{https://www.la-biblia-de-la-paella.es/recette/paella-de-marisco/}{La biblia de la paella} y Chef Amadeo en el canal Paellas y Cocina Valenciana (\href{https://www.youtube.com/watch?v=WbaWsTqg9CY}{paella de mariscos}, \href{https://www.youtube.com/watch?v=rem5NKXjEBo}{fumet de pescado}) \\

Sobre la paella (el sartén): es importante que la altura del arróz no sea de más de \SIrange{2}{3}{cm}. Se necesita un sartén tal que tenga un area de \SIrange{3}{4}{cm^2} por gramo de arroz, y que el fuego sea uniforme en toda la base. Esta receta funciona bien para dos personas usando un sartén normal de \SI{\sim 30}{cm} sobre una ornilla normal grande. Si se quiere escalar se necesitaría una paella de verdad con un horno paellero (o usar leña).\\

\underline{Ingredientes}

\textbf{Fumet (caldo o fondo de pescado)}
\begin{itemize}
\item \SI{2}{litros} de agua
\item \SI{700}{gr} de moralla (e.g. cangrejo, pescado y camarones\footnote{Como cangrejos, pescados pequeños, cabezas o esqueletos de pescado, cabezas de camarón, etc., pero nunca víseras.}
\item 1 poro
\item 1 cebolla pequeña
\item 1 zanahoria
\item 1 bara de apio
\item 1 chile ñora\footnote{Me parece que es el mismo que el bola o cascabel de México, pero no estoy seguro}
\item 1 diente de ajo
\item Una ramita de perejil
\item Aceite de oliva
\end{itemize}

\textbf{Paella}
\begin{itemize}
\item \SI{1}{taza} de arroz bomba\footnote{El arróz bomba absorbe mucha agua y sabores, y al terminar la cocción queda firme y con los granos separados. Puede que otro arroz corto sirva, pero quizá la proporción de agua y el tiempo de cocción sean diferentes.} (\SI{200}{gr})
\item \SI{400}{gr} de mariscos (e.g. pescado y gambas)
\item \SI{3}{tazas} de fumet
\item 20 hebras de azafran
\item \SI{1/2}{taza} de agua
\item \SI{1/2}{cucharadita} de pimentón dulce (aka paprika)
\item \num{1/4} cebolla
\item \num{1/2} tomate
\item 2 \SI{1/2}{cucharadas} de aceite de oliva
\item 1 diente de ajo
\item \SI{2}{cucharaditas} de sal
\end{itemize}

\underline{Instrucciones}

\begin{enumerate}
\item .
\end{enumerate}

