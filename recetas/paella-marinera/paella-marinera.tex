\subsection{Paella marinera}

Basada en: \href{https://www.la-biblia-de-la-paella.es/recette/paella-de-marisco/}{La biblia de la paella} y Chef Amadeo en el canal Paellas y Cocina Valenciana (\href{https://www.youtube.com/watch?v=WbaWsTqg9CY}{paella de mariscos}, \href{https://www.youtube.com/watch?v=rem5NKXjEBo}{fumet de pescado}) \\

Sobre la paella (el sartén): es importante que la altura del arróz no sea de más de \SIrange{2}{3}{cm}. Se necesita un sartén tal que tenga un area de \SIrange{3}{4}{cm^2} por gramo de arroz, y que el fuego sea uniforme en toda la base. Esta receta funciona bien para dos personas usando un sartén normal de \SI{\sim 30}{cm} sobre una ornilla normal grande. Si se quiere escalar se necesitaría una paella de verdad con un horno paellero (o usar leña).\\

\underline{Ingredientes}

\textbf{Fumet (caldo o fondo de pescado)}
\begin{itemize}
\item \SI{2}{litros} de agua
\item \SI{700}{gr} de moralla (e.g. cangrejo, pescado y camarones\footnote{Como cangrejos, pescados pequeños, cabezas o esqueletos de pescado, cabezas de camarón, etc., pero nunca víseras.}
\item 1 poro
\item 1 cebolla pequeña
\item 1 zanahoria
\item 1 bara de apio
\item 1 chile ñora\footnote{Me parece que es el mismo que el bola o cascabel de México, pero no estoy seguro}
\item 1 diente de ajo
\item Una ramita de perejil
\item Aceite de oliva
\end{itemize}

\textbf{Paella}
\begin{itemize}
\item \SI{1}{taza} de arroz bomba\footnote{El arróz bomba absorbe mucha agua y sabores, y al terminar la cocción queda firme y con los granos separados. Puede que otro arroz corto sirva, pero quizá la proporción de agua y el tiempo de cocción sean diferentes.} (\SI{200}{gr})
\item \SI{400}{gr} de mariscos (e.g. pescado y gambas)
\item \SI{3}{tazas} de fumet
\item 20 hebras de azafrán
\item \SI{1/2}{taza} de agua
\item \SI{1/2}{cucharadita} de pimentón dulce en polvo (aka paprika)
\item \num{1/4} cebolla
\item \num{1/2} tomate
\item 2 \SI{1/2}{cucharadas} de aceite de oliva
\item 1 diente de ajo
\item \SI{1}{cucharadita} de sal
\end{itemize}

\underline{Instrucciones}

\textbf{Fumet}
\begin{enumerate}
\item Partir todas las verduras en trozos grandes y sofreir en suficiente aceite con el chile ñora y el ajo machacado.
\item Sofreir la moralla.
\item Si se usan cangrejo o similares, machacar un poco para romper el caparazón.
\item Poner el agua a fuego alto, y cuando esté hirviendo agregar todos los ingredientes
\item Bajar el fuego y cocer por \SI{20}{min} más.
\item Colar
\end{enumerate}

\textbf{Paella}
\begin{enumerate}
\item Triturar el azafrán con un mortero y poner en la \SI{1/2}{taza} de agua por \SI{30}{min}
\item Picar la cebolla y poner a sofreir a fuego medio bajo en el aceite por \SI{\sim 2}{min}.
\item Rallar el jitomate y agregar. Sofreir por \SI{1}{min}.
\item Agregar el pimentón y mezclar.
\item Si se usan camarones o similares, cocer en el sofrito por \SIrange{\sim 30}{60}{s} por lado. Sacar y reservar.
\item Agregar los trozos de pescado y cocer ligeramente.
\item Agregar el fumet y la sal. Subir a fuego alto y esperar a que se halla evaporado \Sim\SI{1/2}{taza} de agua (\SI{\sim 2}{min} si el fumet ya estaba caliente).
\item Agregar el arroz y el agua con el azafrán.
\item El tiempo de cocción total del arroz es de \SI{18}{min}, de la manera siguiente:
\begin{enumerate}
\item A fuego alto los primeros \SI{8}{min}. El agua llega mas o menos a la altura del arroz. 
\item Si se usaron camarones o similares, en este punto se ponen sobre el arroz, presionando ligeramente.
\item A fuego bajo por los siguientes \SI{8}{min}. El agua se ha evaporado casi por completo, sólo queda un capa fina en el fondo.
\item A fuego alto los últimos \SI{2}{min}.
\item Durante los últimos \SIrange{30}{60}{s} se debe de escuchar que se está dorando el arroz del fondo (\textit{socarrat}). Debe de quedar café oscuro mas no quemado. Revisar con una cuchara que se haya formado socarrat en todas partes (la cuchara no patina en el fondo).
\end{enumerate}
\item Dejar reposar por un par de minutos antes de servir.
\end{enumerate}