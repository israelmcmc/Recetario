\subsection{Tacos al vapor}
\label{sec:tacos-al-vapor}

Basada en: \href{https://www.youtube.com/watch?v=kH0nLhOtY2s}{como hacer TACOS DE CANASTA, la receta secreta de los taqueros, \# 456 | Chef Roger} y \href{https://www.youtube.com/watch?v=PoDRiJ3zVmc}{TACOS AL VAPOR ESTILO JALISCO - Alejandra de Nava}\\

La preparación de los guisos y el aceite es igual que los  \hyperref[sec:tacos-de-canasta]{tacos de canasta}, sin embargo el procedimiento para armar los tacos es diferente. La principal diferencia es que en los tacos de canasta todo tiene que estar caliente al poner en la canasta ya que generan su propio vapor. En los tacos al vapor no es indispensable que los tacos estén calientes ya que se usa una vaporera.\\

\underline{Ingredientes}\\
Ver receta de \hyperref[sec:tacos-de-canasta]{tacos de canasta}\\

\underline{Instrucciones}

\begin{enumerate}
\item Pasar la tortilla por aceite por ambos lados\footnote{Es indispensable que estén completamente pasadas por aceite para que no se rompan, no trates de hacer light. De preferencia usar tortillas recién hechas para que no se rompan.}. 
\item Poner guiso y cerrar.
\item Acomodar en la vaporera capa por capa, tratando de dejar un hoyo en el centro por donde pueda circular el vapor. La primer y última capa són sólo de tortillas entendidas, también pasadas por aceite, para proteger los tacos.
\item Tapas con una toalla y luego con la tapadera.
\item Prender la vaporera y apagar 15 min después de que empezó a salir el vapor.
\item Dejar reposar por 15-30 min.
\end{enumerate}

