\subsection{Chiles rellenos}
\textbf{Ingredientes}
\begin{itemize}
\item 1 chile poblano
\item 2 huevos
\item $\sim$100 gr de queso para derretir (e.g. Oaxaca, Chihuahua, asadero)
\item Harina (para espolvorear)
\item Dos cucharadas de aceite
\end{itemize}

\textbf{Instrucciones}
\begin{enumerate}
\item Poner el chile al fuego hasta que este blandito y la mayor parte de la piel este quemada.
\item Envolverlo con pl\'astico y dejar que se enfr\'ie.
\item Remover la piel.
\item Hacer una abertura por un lado y rellenar de queso.
\item Espolvorear y cubrir con harina.
\item Batir las dos claras de huevo hasta punto de turron. Cuando este listo, agregar media yema y batir. Descartar el resto.
\item Cubrir con el turron y fre\'ir. Se empieza por donde esta la abertura pasa sellarlo. Con una espatula poner m\'as turron si se descubri\'o alguna parte.
\end{enumerate}

\textbf{Ingredientes salsa}
\begin{itemize}
\item 4 jitomates cocidos
\item 2 tazas de agua
\item 1 cuadrito de caldo de pollo (11 gr)
\item \sfrac{1}{4} cuarto de cebolla.
\item 1 cucharadita de ajo molido
\item 1 cucharadita de sal
\item 1 cuacharadita de oregano molido
\end{itemize}
