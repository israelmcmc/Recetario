\subsection{Conos de hojaldre}
\label{sec:conos-hojaldre}

También llamados caracoles, cuernitos, canutillos o tornillos, rellenos de crema pastelera. \\

Receta para 12 conos de \SI{5}{pulgadas}.\\

\underline{Ingredientes}

\begin{itemize}
\item \SI{450}{gr} de pasta de hojaldre
\item \num{\sim 1}\SI{1/2}{tazas} de crema pastelera
\item 1 huevo batido
\item Azúcar para espolvorear
\end{itemize}

\underline{Instrucciones}

\begin{enumerate}
\item Con un rodillo extender la masa hasta formar un cuadrado de aproximadamente \SI{50}{cm} por lado \footnote{Si se van a hacer por partes, mantén en el refrigerador la porción de masa que no se va a usar al momento. Si no se derrite la mantequilla y no se alza.}. Cortar orillas.
\item Dividir en 12 tiras.
\item Con una brocha poner una capa fina de huevo en una orilla, que cubra la mitad de la tira a lo largo. Esto es para que se pegue bien la masa.
\item Enroscar en el cono de metal, cuidando de que la parte con huevo quede sobre la masa y no sobre el cono de metal. Apretar al inicio y al final para sellar.
\item Pintar con huevo y espolvorear azúcar.
\item Hornear a \SI{200}{C} (\SI{400}{F}) por \SI{20}{min}, o hasta que todos los conos esten bien doraditos.
\item Remover con un giro el cono de metal cuando estén tibios.
\item Rellenar con la crema pastelera.
\end{enumerate}

