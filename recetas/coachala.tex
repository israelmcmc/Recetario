\subsection{Coachala}

Fuente: recetario abue\\

\underline{Ingredientes}

\begin{itemize}
\item 1 pollo entero \footnote{Originalmente se hacía con una gallina}
\item \sfrac{1}{2} kilo de masa
\item 100 gr de chile ancho (sin semilla)
\item 100 gr de manteca
\item \sfrac{1}{4} kg de tomate verde
\item 6 dientes de ajo
\item Sal al gusto (\Sim 2-4 cucharaditas)
\end{itemize}

\underline{Instrucciones}

\textbf{Pollo}
\begin{enumerate}
\item Cocer el pollo(\Sim 20 en olla a presión). Reservar el caldo.
\item Desmunzar la pechuga, piernas, alas y muslos.
\item Licuar el resto del pollo (¡sin los huesos!) con un poco del caldo.
\end{enumerate}

\textbf{Guiso}
\begin{enumerate}
\item Dorar los chiles por unos segundos en la manteca. Reservar la manteca requemada.
\item Cocer lo tomates hasta que revienten.
\item Licuar los tomates, el ajo, los chiles y un poco de caldo.
\item Colar la salsa
\item Añadir agua/caldo a la masa para hacer un atole y llevar a hervor, meneando todo el tiempo.
\item Añadir a la salsa la manteca requemada, la sal y la salsa, y llevar a hervor, meneando todo el tiempo.
\end{enumerate}

\textbf{Final}
\begin{enumerate}
\item En una olla aparte vaciar el pollo desmenuzado, el pollo molido y el guiso. Dependiendo del tamaño del pollo quizá no todo el guiso sea necesario.
\item Dejar hervir, meneando, hasta que adquiera la consistencia adecuada. Si es necesario agregar más caldo/agua.
\item Servir con tostadas y cebolla desflemada con limón.
\end{enumerate}