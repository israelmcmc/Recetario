\subsection{Tacos al pastor}

Basada en: \href{https://www.youtube.com/watch?v=W03EIyBCOPI&list=LL70vwP7i8PV9gx3t6SvBbRQ&index=598}{Luisito informa - Un día con un taquero}, \href{https://www.youtube.com/watch?v=F7N3l6XGn-0}{Al estilo del Chef Roger - Carne al pastor},
\href{https://www.youtube.com/watch?v=pmzl92R3rJk}{Toque y sazón - Tacos al pastor}\\

\underline{Ingredientes}

\begin{itemize}
\item \SI{10}{kg} de cabeza de lomo de cerdo \footnote{En inglés se llama ``Boston butt'', aunque este incluye también una parte del omóplato y carne alrededor.}
\item 150-200 tortillas (\num{\sim 6}-8 kg de masa)
\item 35 chiles guajillo
\item 20 chiles anchos
\item 5 chiles chipotles (sin adobar)
\item \SI{1}{taza} de jugo de naranja agria \footnote{Sustituir con jugo de naranja y 3 limones}
\item 1\SI{1/2}{taza} de vinagre
\item 10 dientes de ajo
\item 20 clavos de olor
\item 1\SI{1/4}{cucharaditas} de comino
\item 1\SI{1/4}{cucharaditas} de oregano
\item 5 cucharaditas de sal
\item 20 pimientas gordas
\item \SI{60}{gr} de paste de achiote
\item Sal extra para la carne
\item 1 cebolla grande para la base
\item 1 piña para coronar
\end{itemize}

\underline{Instrucciones}
\begin{itemize}
\item Quitar semillas y venas de los chiles secos.
\item Poner en agua hirviendo por \SI{\sim 5}{min} para que se ablanden.
\item Licuar con el resto de lo incredientes del adobo
\item Colar el adobo y descartar resto.
\item Cortar la carne en filetes de \SI{\sim 7}{mm} ($\sim$ medio dedo de ancho).
\item Espolvorear caa filete con sal por ambos lados
\item Revolver con el adobo. Dejar marinar por lo menos \SI{2}{hr}.
\end{itemize}

\underline{Notas}
\begin{itemize}
\item Para armar el trompo se compienza con un cebolla grande para que el cuchillo no toque el metal. Después de ensartar los filetes las orillas de los filetes se doblan hacia andentro para darle forma. Se van cortando pedazos sobrantes y poniendo encima. Intercalar filetes grandes con pedazos chicos.
\item Después de \SI{\sim 15}{min} en el fuego, con el cuchillo se le termina de dar forma al trompo para que queden las paredes lisas. Es normal que una buena parte de los pedazos que salgan sigan algo crudos, así que esta primera tanda se termina de cocinar en sartén.
\item El cuello de botella es que tan rápido se dora el exterior del trompo. Por esto, usar la mayor cantidad de carbón posible y hacer el trompo lo más alto posible para aprovechar toda la columna de carbón.
\item Es crucial tener un cuchillo afilado y grande. Para el corte el cuchilo entra casi vertical, con tajadas en una sola dirección hacia uno, y tratando de que el corte no sea hacia los lados para no girar el trompo. El trompo bloquea el fuego de las manos para no quemarse. El corte debe de ser lo más delgado posible y salir laminado.
\end{itemize}
