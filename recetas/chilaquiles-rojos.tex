\subsection{Chilaquiles rojos}\label{chilaquiles-rojos}

Basada en: \href{https://www.youtube.com/watch?v=I8xJOKFjDF4}{CHILAQUILES ROJOS - Vicky Receta Facil} \\

\underline{Ingredientes}

\textbf{Salsa}
\begin{itemize}
\item 4 jitomates
\item 3 chiles huajillo
\item Chile de arbol al gusto
\item Sal al gusto
\item 1 diente de ajo
\item 1 rebanada de cebolla
\item 1 cubo de consome de pollo
\end{itemize}

\textbf{Totopos}
\begin{itemize}
\item $\sfrac{1}{2}$ kg de tortilla
\item Aceite vegetal
\end{itemize}

\textbf{Para acompañar}
\begin{itemize}
\item Crema
\item Queso seco
\item Cebolla con limon
\end{itemize}

\underline{Instrucciones}

\textbf{Salsa}
\begin{enumerate}
\item Cocer los tomates hasta que reviente
\item Remover las semillas del chile huajillo. 
\item Cocer todos los chiles hasta que se aguaden.
\item Licuar todo
\item Hervir por \Sim10-15min. Agregar agua si es necesario.
\end{enumerate}

\textbf{Totopos}
\begin{itemize}
\item Dejar las tortillas afuera por \Sim 1dia para que se sequen
\item Cortar en 6
\item Poner en aceite caliente hasta que se doren.
\item Dejar escurrir.
\end{itemize}

\textbf{Chilaquiles}
\begin{itemize}
\item Hacer la salsa.
\item Hacer los totopos. De preferencia hacerlos hasta que ya este lista la salsa.
\item Bañar los totopos con las salsa y revolver bien en un sartén.
\item Cubrir con crema, queso y cebolla desflemada. 
\end{itemize}

