\subsection{Masa de maíz}
\label{receta:masa-maiz}

Basada en \href{https://mexicotierraviva.org/con-maiz-masa-de-nixtamal/}{receta para masa de nixtamal de México Tierra Viva}\\

\underline{Ingredientes} (\SI{\sim 1}{kg} de masa, o \num{\sim 20} tortillas)
\begin{itemize}
\item \SI{1/2}{kg} de granos de maiz seco
\item \SI{1/2}{litro} de agua
\item \SI{1}{cucharada} de cal
\end{itemize}

\underline{Instrucciones}
\begin{enumerate}
\item Dejar remojando el maíz por una noche o al menos un par de horas
\item Escurrir
\item Diluir la cal en el agua
\item Agregar el maiz
\item Calentar a fuego alto y bajar el fuego cuando comience a hervir.
\item Cocer por 20 min
\item Dejar reposar tapado hasta que enfríe (\SI{\sim 5}{hr}).
\item Lavar varias veces, frotando los granos unos con otros, hasta que salga clara el agua (\num{\sim 4} cambios de agua)
\item Pasar por el molino 2 veces.
\item Agregar un poco de agua, la necesaria, y amasar.
\end{enumerate}

\underline{Notas}
\begin{itemize}
\item Varios de estos pasos no son completamente necesarios, pero mejoran la consistencia de la masa. Una masa más elástica y fácil de trabajar resulta en tortillas que se inflan mejor y son menos quebradizas. No sea huevón.
\begin{itemize}
\item El remojado no es del todo necesario, pero ayuda a mejorar la consistencia de la masa.
\item El mínimo tiempo de cocción necesario es hasta que el pellejito (pericarpio) se cae, apriximadamente \SI{10}{min}. Sin embargo, cocerlo por más tiempo mejora la consistencia de la masa. Granos grandes quizá necesiten hasta 30 min de cocción. Los granos deben de estar cocidos por fuera, con el centro todavía crudo y nunca ya reventados.
\item \SI{\sim 30}{min} de reposo después de la cocción es suficiente, pero la consistencia de la masa mejora considerablemente si se deja más tiempo.
\item Pasarlo por el molino una segunda vez puede parecer innecesario, especialmente considerando que es difícil y requiere de hacer presión. Sin embargo, la consistencia de la masa después de la segunda pasada es considerablemente mejor. Otra opción, que he visto en pueblos, es dándole una pasada rápida por metate después del molino. La masa debe de quedar suave, homogenea y sin pedacitos duros.
\end{itemize}
\item Para tortillas es mejor agregar bastante agua al final. El limitante es que la masa no se pegue a las manos y la tortilla cruda se pueda manipular sin romperse.
\end{itemize}