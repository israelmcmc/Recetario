\subsection{Masa de maíz nixtamalizado}
\label{receta:masa-maiz}

Basada en \href{https://mexicotierraviva.org/con-maiz-masa-de-nixtamal/}{receta para masa de nixtamal de México Tierra Viva}\\

\underline{Ingredientes} (\SI{\sim 1}{kg} de masa, o \num{\sim 20} tortillas)
\begin{itemize}
\item \SI{1}{kg} de granos de maiz seco
\item 2\SI{1/2}{litro} de agua
\item 2\SI{1/2}{cucharada} de cal
\item \SI{\sim 300}{ml} de agua extra
\end{itemize}

\underline{Instrucciones}
\begin{enumerate}
\item Diluir la cal en el agua y gregar el maiz
\item Calentar a fuego alto hasta que de un hervor\footnote{Hervirlo por unos cuantos minutos más esta bien, aunque no es necesario. Si se coce por mucho que ``apozola'' y ya no jala bien.}.
\item Dejar reposar por \SI{\sim 8}{hr}.
\item Lavar varias veces, frotando los granos unos con otros, hasta que salga clara el agua (\num{\sim 4} cambios de agua)
\item Moler lo más fino posible
\item Agregar el agua y amasar. Eentre menos seca mejor pero sin que se pegue a las manos torteando. 
\end{enumerate}
