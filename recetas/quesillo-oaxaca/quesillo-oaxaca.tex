\subsection{Quesillo de Oaxaca}

Basada en : \href{https://cheesemaking.com/products/queso-oaxaca-recipe}{Queso Oaxaca Recipe - New England Cheese Making} y Los quesos mexicanos tradicionales, de Villegas, et al. \\

\underline{Ingredientes}
\begin{itemize}
\item \SI{2}{galones} de leche bronca \footnote{No necesita pasteurizarse, ya que durante el proceso se lleva la temperatura del cuajo a 80. Leche pasteurizada (en especial a bajas temperaturas) también podría funcionar usando al iniciador y la cloruro de calcio. La leche ultra-pasteurizada convencional nunca jala.}
\item \num{1/4} de pastilla para cuajar \footnote{Esto es para la marca Walcoren. Necesita la mitad de cuajo de lo que normalmente se necesitaría si no se acidificara tanto la leche.} 
\item Opcional: \SI{1/2}{cucharadita} de iniciador de fermentación termofílico \footnote{Esto es para el Thermo B Starter Culture. Seguir indicaciónes para otra marca. Se necesita el mismo tipo que para hacer mozzarella. Si la leche no esta pasteurizada, se puede usar le método tradicional y no se necesita.}
\item Opcional: \SI{1} cucharadita de cloruro de calcio al 32\% en peso \footnote{Para leche pasteurizada, no se necesita para leche bronca.}
\item Salmuera
\begin{itemize}
\item \SI{1}{galon} de agua
\item 2\SI{1/4}{de} de cucharada de sal
\item \SI{1}{cucharada} de solución de cloruro de calcio
\item \SI{1}{cucharada} de vinagre blanco
\end{itemize}
\end{itemize}


\underline{Instrucciones}
\begin{enumerate}
\item Llevar la leche a \SI{35}{C} (\SI{94}{F})
\item Acidificar la leche hasta \SIrange{30}{32}{D} \footnote{Usar titulación para medir los grados Dornic. 1) Mezclar \SI{9}{ml} de leche o suero con 5 gotas de fenolftaleína 2) Agregar gota a gota una solución de 0.1N NaOH) Los grados Dornic son $10\times$ el volumen necesario en ml de 0.1N NaOH para que se vuelva rosa por \SIrange{10}{20}{s}.} 
\begin{itemize}
\item \textbf{Metodo tradicional}. Si se usa leche bronca simplemente se puede dejar reposar (lejos de fuentes de calor, como el piloto de una estufa) por \SIrange{8}{24}{hr} (dependiento de la temperatura del ambiente) hasta que alcance el nivel de acidez necesario.
\item \textbf{Con iniciador}. Espolvorear el iniciador en la superficie y dejar que se humedezca por \SI{2}{min}. Revolver. Dejar reposar por \SI{\sim 3}{hr}.
\end{itemize}
\end{enumerate}
