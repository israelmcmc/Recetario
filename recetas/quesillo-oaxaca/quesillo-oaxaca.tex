\subsection{Quesillo de Oaxaca}

Basada en : \href{https://cheesemaking.com/products/queso-oaxaca-recipe}{Queso Oaxaca Recipe - New England Cheese Making}, \href{http://www.sagarpa.gob.mx/desarrolloRural/Documents/fichasaapt/Elaboraci%C3%B3n%20de%20quesos.pdf}{SAGARPA - Elaboración de quesos tipo Panela y Oaxaca} y Los quesos mexicanos tradicionales, de Villegas, et al. \\

\underline{Ingredientes}
\begin{itemize}
\item \SI{2}{galones} de leche \footnote{De preferencia bronca. No necesita pasteurizarse, ya que durante el proceso se lleva la temperatura del cuajo a \SI{80}{C}. Leche pasteurizada (en especial a bajas temperaturas, sin homogeneizar) también podría funcionar si se inocula primero la leche y se usa la cloruro de calcio. La leche ultra-pasteurizada convencional nunca jala.}
\item \num{1/2} de pastilla para cuajar \footnote{Esto es para la marca Walcoren.} 
\item \SIrange{20}{30}{gr} de sal
\item Opcional: \SI{1/2}{cucharadita} de inoculador termofílico \footnote{Esto es para el Thermo B Starter Culture. Seguir indicaciónes para otra marca. Se necesita el mismo tipo que para hacer mozzarella. Si la leche no esta pasteurizada, se puede usar el método tradicional y no se necesita, aunque es más rapido.}
\item Opcional: \SI{1} cucharadita de cloruro de calcio al 32\% en peso \footnote{Para leche pasteurizada, no se necesita para leche bronca.}
\end{itemize}


\underline{Instrucciones}
\begin{enumerate}
\item Llevar la leche a \SI{35}{C} (\SI{94}{F})
\item Acidificar la leche \footnote{A \SIrange{30}{32}{D}. Los grados Dornic se pueden medir mediante titulación, pero no es necesario.} 
\begin{itemize}
\item \textbf{Metodo tradicional}. Si se usa leche bronca simplemente se puede dejar reposar por \SIrange{8}{24}{hr}, dependiento de la temperatura del ambiente, hasta que alcance el nivel de acidez necesario.
\item \textbf{Por inoculación}. Espolvorear el inoculador en la superficie y dejar que se humedezca por \SI{2}{min}. Revolver. Reposar por \SI{\sim 3}{hr}\footnote{En lugares fríos quizá sea necesario mantener la temperatura a \SI{\sim 35}{C} mediante baño María durante todo el proceso}.
\end{itemize}
\item Diluir la pastilla de cuajo y agregar.
\item Dejar reposar por \SIrange{30}{60}{min}, hasta que se pueda cortar con un cuchillo y salga limpio.
\item Cortar cuadrados de \SI{\sim 1}{pulgada}. Dejar reposar por \SI{\sim 10}{min}.
\item Batir para moler los granos ligeramente. Dejar reposar por \SI{\sim 5}{min}.
\item Remover el suero con un colador, sin apretar.
\item Dejar reposar por \SIrange{2}{5}{hr} en baño maría a \SI{\sim 45}{C} (\SI{110}{F}) hasta que el suero se acidifique \footnote{\SIrange{32}{35}{D}} y pase la prueba de elasiticidad (un fragmento de cuajada se pone en agua a \SI{80}{C} (\SI{175}{F}) por \SIrange{5}{10}{min}, si se estira ya es el punto).
\item Rebanar el cuajo en tiras y poner en agua a \SI{80}{C} (\SI{175}{F}) por  \SIrange{5}{10}{min}.
\item Estirar el cuajo y doblarlo sobre sí mismo \num{\sim 20} veces. Poner el queso en el agua caliente (que debe de mantenerse a \SI{\sim 80}{C}) después de cada ciertos dobleces, cuando ofrezca resistencia. Meter las manos en agua con hielo periodicamente para no quemarse.
\item Meter al agua caliente por última vez y comenzar a estirar desde un extremo formando una tira larga delgada, de preferencia plana\footnote{Una alternativa es ponerlos en una mesa y rociarlos con agua fría. Así quedan las tiras quedan planas como un liston.}, que se va introduciendo en el agua fría con hielos conforme va saliendo. 
\item Orear por \SIrange{20}{30}{min}.
\item Frotar con la sal usando 3\% del peso final del queso.
\item Formar las bolas. Envolver en plástico y dejar reposar unas horas para que absorba la sal.
\end{enumerate}
