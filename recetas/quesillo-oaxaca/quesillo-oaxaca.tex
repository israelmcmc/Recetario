\subsection{Quesillo de Oaxaca}

Basada en : \href{https://cheesemaking.com/products/queso-oaxaca-recipe}{Queso Oaxaca Recipe - New England Cheese Making},  \href{http://www.sagarpa.gob.mx/desarrolloRural/Documents/fichasaapt/Elaboraci%C3%B3n%20de%20quesos.pdf}{SAGARPA - Elaboración de quesos tipo Panela y Oaxaca} y Los quesos mexicanos tradicionales, de Villegas, et al. \\

\underline{Ingredientes}
\begin{itemize}
\item \SI{2}{galones} de leche bronca \footnote{No necesita pasteurizarse, ya que durante el proceso se lleva la temperatura del cuajo a 80. Leche pasteurizada (en especial a bajas temperaturas) también podría funcionar usando al iniciador y la cloruro de calcio. La leche ultra-pasteurizada convencional nunca jala.}
\item \num{1/4} de pastilla para cuajar \footnote{Esto es para la marca Walcoren. Necesita la mitad de cuajo de lo que normalmente se necesitaría si no se acidificara tanto la leche.} 
\item Opcional: \SI{1/2}{cucharadita} de iniciador de fermentación termofílico \footnote{Esto es para el Thermo B Starter Culture. Seguir indicaciónes para otra marca. Se necesita el mismo tipo que para hacer mozzarella. Si la leche no esta pasteurizada, se puede usar le método tradicional y no se necesita.}
\item Opcional: \SI{1} cucharadita de cloruro de calcio al 32\% en peso \footnote{Para leche pasteurizada, no se necesita para leche bronca.}
\item Salmuera
\begin{itemize}
\item \SI{1}{galon} de agua
\item 2\SI{1/4}{de} de cucharada de sal
\item \SI{1}{cucharada} de solución de cloruro de calcio
\item \SI{1}{cucharada} de vinagre blanco
\end{itemize}
\end{itemize}


\underline{Instrucciones}
\begin{enumerate}
\item 
\end{enumerate}
