\subsection{Palak Paneer}

Basada en : \href{https://www.vahrehvah.com/palak-paneer}{Palak paneer - Vah Reh Vah} \\

\underline{Ingredientes}
\begin{itemize}
\item $\sfrac{1}{4}$ kg de espinacas
\item $\sfrac{1}{4}$  de cebolla picada
\item 1 chile serrano picado
\item $\sfrac{1}{4}$ cucharaditas de chile en polvo
\item $\sfrac{1}{2}$ cucharaditas de pasta de gengibre
\item 1 tomate picado
\item Sal al gusto
\item $\sfrac{1}{2}$ cucharadita de colantro seco
\item $\sfrac{1}{2}$ cucharadita de comino molido
\item $\sfrac{1}{2}$ cucharadita de garam masala
\item $\sfrac{1}{2}$ cucharadita de alholva (fenogreco)
\item 3 dientes de ajo molidos
\item $\sfrac{1}{8}$ de cucharaditas de curcuma (turmeric)
\item 150 gr de paneer en pedacitos de \Sim 1cm
\item 2 cucharadas de aceite
\end{itemize}


\underline{Instrucciones}
\begin{enumerate}
\item Lavar las espinacas y ponerla en agua hirviendo por dos minutos. Desechar agua.
\item Licuar la espinaca con agua y reservar.
\item Calentar el aceite a fuego bajo y agregar el ajo. Si se usa garam masala entera agregarla en este paso (y remover después).
\item Agregar la cebolla y un poco de sal. Freir hasta que este café.
\item Agregar el tomate y freir hasta que se haga pasta.
\item Agregar todas las especias.
\item Agregar la esponaca molida y dejar hervir hasta que tenga unas consistencia cremosa.
\item Agregar el paneer y dejar por unos minutos.
\end{enumerate}
