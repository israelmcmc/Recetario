\subsection{Arroz a la mexicana}
\textbf{Ingredientes}
\begin{itemize}
\item 2 tomates medianos ($\sim$400gr)
\item 2 cucharaditas de aceite
\item 1 taza de arroz
\item 2 tazas de agua
\item Cubo de caldo de pollo (11gr)
\item Cuarto de cebolla ($\sim$ 70gr)
\item \sfrac{3}{4} de cucharadita de sal (o al gusto)
\item \sfrac{3}{4} de cucharadita de ajo en polvo (o al gusto)
\item Verdura cocida picada al gusto (zanahoria, elote o chicharo)
\end{itemize}

\textbf{Instrucciones}
\begin{enumerate}
\item Se hierven los tomates hasta que la cáscara se abra ($\sim$ 10min), se pelan y muelen.
\item \textit{Opcional}: Se lava el arroz hasta que el agua salga relativamente clara ($\sim$ 3 veces).
\item Se pone a hervir las dos tazas de agua y se disuelve el cubo de pollo.
\item Se precaliente un sartén con el aceite y se añade el arroz. Se mantiene en movimiento hasta que el arroz este amarillo. Se agrega la cebolla picada y se continua moviendo hasta que tenda un color dorado sin quemarse.
\item Se agrega y revuelve en el sart\'en el tomate, la sal, el ajo y el agua con el cubo de pollo.
\item Se deja a fuego lento, hasta que toda el agua se haya evaporado pero antes de que se pegue el arroz. Agregar las verduras un poco antes de esto si se desea.
\end{enumerate}
