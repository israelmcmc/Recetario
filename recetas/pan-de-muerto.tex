\subsection{Pan de muerto}

Basada en : \href{https://www.youtube.com/watch?v=PnYVBsY-xAk}{Yuri de Gortari - Pan de Muerto}\\

\underline{Ingredientes}

\textbf{Masa}
\begin{itemize}
\item 90 gr de mantequilla
\item 2 huevos
\item 200 ml de leche tibia
\item 150 gr de azucar
\item 1 naranja mediana
\item 15 gr de levadura
\item 1 cucharaditas de vainilla
\item \sfrac{1}{2} kg de harina de trigo
\end{itemize}

\underline{Instrucciones}
\begin{enumerate}
\item Poner la levadura en leche tibia, con una cucharada de azucar y una de harina, haciendo una especie de atolito. Dejar fermentar en un ambiente caliente, (e.g. sobre el horno), hasta que duplique su tamaño.
\item Formar una fuente con sal, el resto del harina, ralladura de cascara de naranja, mantequilla blanda, azucar y se amasa todo hasta integrar.
\item Agregar huevos y vainilla (o agua de azahar) y amasar hasta que la masa se despegue.
\item Incorporar levadura fermentada a la masa y mezclar hasta obtener una mezcla homogenea.
\item Dejar reposar masa en un bowl ligeramende engrasado hasta que duplique su tamaño (\Sim 1 hr).
\item Amasar y separar en bolitas (\Sim 6) y redondear a mano, colocandolos en una charola engrasada.
\item Formar huesitos con tiras y colocar sobre el pan.
\item Barnizar con una capa delgada de mantequilla derretida.
\item Espolvorear azucar.
\item Dejarla reposar por \Sim 30 min hasta que dupliquen su tamaño. 
\item Hornear a 180C (350F) por \Sim 20 min, hasta que se empiece a dorar la superficie.
\end{enumerate}
