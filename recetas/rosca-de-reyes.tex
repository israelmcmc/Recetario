\subsection{Rosca de Reyes}

Basada en : \href{https://www.youtube.com/watch?v=4IgNDiX7cHg}{Vicky Receta Facil - Rosca de Reyes}. Yo uso un huevo extra.\\

\underline{Ingredientes}

\textbf{Masa}
\begin{itemize}
\item 135 gr de mantequilla
\item 4 huevos
\item \sfrac{1}{2} taza de leche tibia
\item \sfrac{3}{4} de taza de azucar
\item 1 cucharadita de sal
\item 11 gr de levadura
\item 2 cucharaditas de vainilla
\item \sfrac{1}{2} kg de harina
\end{itemize}

\textbf{Costra}
\begin{itemize}
\item 100 gr de manteca vegetal
\item 100 gr de harina
\item 100 gr de azucar glass
\item 2 yemas de huevo
\end{itemize}

\textbf{Extras}
\begin{itemize}
\item Ate (e.g. mebrillo, guayaba, tecojote)
\item Higos cristalizados
\item Cerezas en almibar
\item Azucar
\item 1 huevo
\item Monitos
\end{itemize}

\underline{Instrucciones}
\begin{enumerate}
\item Poner la levadura en leche tibia, con una cucharada de azucar y una de harina. Dejar deposar hasta que las burbujas dupliquen el tamaño.
\item Amasar el resto de los ingrediente de la masa hasta que despegue.
\item Dejar reposar cubiertas hasta que duplique su tamaño (\Sim 1.5 hr).
\item En una superficie harinada hacer una tira de 1m de largo.
\item Meter monitos por abajo.
\item Hacer la rosca en una charola engrasada. Las puntas se unen haciendo un hueco de un lado y un pico del otro.
\item Mezclar lo ingredientes de la costra hasta que este homogenea.
\item Barnizar con una capa delgada de huevo
\item Poner costa y dulces. Espolvorear azucar en la costra.
\item Dejarla reposar por 30 min.
\item Hornear a 180C (350F) por 20 min.
\end{enumerate}
