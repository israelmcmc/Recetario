\subsection{Carnitas}

Basada en: \href{https://www.youtube.com/watch?v=RHBXS7Oo1VM}{La ruta del sabor - Carnitas estilo Michoacan} y \href{https://www.youtube.com/watch?v=3Yz4mSFP9G8}{Carnitas estilo michoacan - Chef-roger style}. \\

\underline{Ingredientes}

\begin{itemize}
\item \SI{1}{kg} de maciza (en pedazos grandes)
\item \SI{1}{kg} de buche
\item \SI{1/2}{kg} de cuerito
\item 2\SI{1/2}{kg} de manteca
\item \SI{1/2}{litro} de agua
\item \SI{1/2}{litro} de agua con una cucharada de sal (salmuera)
\end{itemize}

\underline{Instrucciones}

\begin{enumerate}
\item Calentar la manteca a fuego alto hasta que este bien caliente. Prueba con un pedazo de cuero primero, debe de burbujear fuertemente al meterlo si esta suficientemente caliente.
\item Agregar la maciza y deja hasta que este dorada por fuera (\SI{\sim10}{min}). Mover ligeramente para que se doren parejo y no se quemen.
\item Agregar el buche y dejar dorar \SI{\sim1}{min}.
\item Agregar el agua. En este punto la manteca ya no debe de estan can caliente.
\item Dejar a fuego medio por \SI{1}{hr}, moviendo ocasionalmente. 
\item Poner los cueritos en la superficie, tapando la carne.
\item Agregar la salmuera.
\item Dejar a fuego medio por \SIrange{\sim1}{1.5}{hr}, moviendo ocasionalmente. Comenzar a checar el cuerito a la hora y apagar cuando este suave.  
\end{enumerate}
