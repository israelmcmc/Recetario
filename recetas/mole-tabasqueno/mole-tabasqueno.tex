\subsection{Mole tabasqueño}

Receta de la Sra. Meraris (mamá de Kris)

%Lo siguiente es para medio pollo sancochado: 2 chiles de color, 2 chiles pasilla, 2 chiles mulato, un cuarto de cebolla morada, 2 tomates, 2 ajos chicos, 3 cucharada de ajonjolí, 10 almendras, una puntita de cuchara de oregano, 3 clavitos, 3 pimientas, un puño de galleta de animalito.

%Los chiles se abren, se les quita la semilla y se ponen a hervir en un traste con agua unos 5 minutos, se baja y se ponen los chiles a escurrir. Luego se pone a tostar el ajonjolí a flama bajita, moviéndolos hasta que se empiecen a dorar, luego se bajan y se licuan en seco y los dejas en un trastecito. La almendra se pone a hervir en un traste con poquita agua, unos 5 minutos, se bajan y se pelan, se licua en seco y se deja en espera.

%Las galletas se fiien en un sarten con aceite, ya que esta dorada se bajan y se deja a espera. La cebolla y el tomate se cortan en rodajas y se sofríen, primero la cebolla y después se pone el tomate y luego los ajos, se baja y se deja en espera.

%En un molcajeta se muele pimienta, oregano y clavito y se ponen a freir junto con los tomates, la cebolla y el ajo.
%Los chiles ya escurridos se ponen a freír, pero solo una pasadita en aceite; porque sino  se amargan.

%Ahora se va licuando todo, un poco de cada cosa para que se incorpore con el caldo de lo sancochado. Se va colando y se pone en un traste extendido y hondito con aceite y ahí se pone lo licuado, ya colado.

%Tiene que quedar aguadito, pero con consistencia, porque al hervir espesa otro poquito más.

%Se le prueba de sal y si le gusta dulce el mole se le pone azúcar. Buen provecho.

\underline{Ingredientes}
\begin{itemize}
\item \num{1/2} pollo
\item 2 chiles de guajillo \footnote{De color = Guajillo?}
\item 2 chiles pasilla
\item 2 chiles mulato
\item \num{1/4} de cebolla morada
\item 2 jitomates
\item 2 dientes ajos chicos \footnote{Dientes o ajos completos}
\item 3 cucharada de ajonjolí
\item 10 almendras
\item \num{1/4} de cucharadita de oregano
\item 3 clavos
\item 3 pimientas gordas
\item un puño de galleta (de animalito o María) \footnote{gr?}
\end{itemize}

\underline{Instrucciones}
\begin{enumerate}
\item 
\end{enumerate}