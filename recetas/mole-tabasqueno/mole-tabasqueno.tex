\subsection{Mole tabasqueño}

Receta ponderada de la Sra. Meraris y Sra. Leti (mamá y tía de Kris)

%Lo siguiente es para medio pollo sancochado: 2 chiles de color, 2 chiles pasilla, 2 chiles mulato, un cuarto de cebolla morada, 2 tomates, 2 ajos chicos, 3 cucharada de ajonjolí, 10 almendras, una puntita de cuchara de oregano, 3 clavitos, 3 pimientas, un puño de galleta de animalito.

%Los chiles se abren, se les quita la semilla y se ponen a hervir en un traste con agua unos 5 minutos, se baja y se ponen los chiles a escurrir. Luego se pone a tostar el ajonjolí a flama bajita, moviéndolos hasta que se empiecen a dorar, luego se bajan y se licuan en seco y los dejas en un trastecito. La almendra se pone a hervir en un traste con poquita agua, unos 5 minutos, se bajan y se pelan, se licua en seco y se deja en espera.

%Las galletas se fiien en un sarten con aceite, ya que esta dorada se bajan y se deja a espera. La cebolla y el tomate se cortan en rodajas y se sofríen, primero la cebolla y después se pone el tomate y luego los ajos, se baja y se deja en espera.

%En un molcajeta se muele pimienta, oregano y clavito y se ponen a freir junto con los tomates, la cebolla y el ajo.

%Los chiles ya escurridos se ponen a freír, pero solo una pasadita en aceite; porque sino  se amargan.

%Ahora se va licuando todo, un poco de cada cosa para que se incorpore con el caldo de lo sancochado. Se va colando y se pone en un traste extendido y hondito con aceite y ahí se pone lo licuado, ya colado.

%Tiene que quedar aguadito, pero con consistencia, porque al hervir espesa otro poquito más.

%Se le prueba de sal y si le gusta dulce el mole se le pone azúcar. Buen provecho.

\underline{Ingredientes}
\begin{itemize}
\item 1 pollo
\item 4 chiles de anchos
\item 4 chiles pasilla
\item 4 chiles mulato
\item \num{1/2} de cebolla morada
\item 3 jitomates chicos maduros
\item 5 dientes ajos chicos
\item 4 cucharada de ajonjolí
\item \SI{50}{gr} almendras \footnote{La receta dice 20, creemos que más es mejor}
\item \SI{\sim 50}{gr} de pasitas \footnote{La receta decía 10 pesos (!), asi que hay que ver}
\item \num{1/2} de cucharadita de oregano
\item 6 clavos
\item 6 pimientas gordas
\item \SI{50}{gr} de galletas María o de animalitos
\item Aceite
\item Sal (\SI{\sim 4}{cucharaditas})
\item \SI{110}{gr} de piloncillo
\end{itemize}

\underline{Instrucciones}
\begin{enumerate}
\item Cocer el pollo por \SI{1}{hr} y reservar el caldo
\item Quitarle las semillas a los chiles, cocer por \SI{5}{min} y escurrir.
\item Cocer las almendras por \SI{5}{min} y pelarlas
\item Dorar el ajonjolí a fuego bajo hasta que empiecen a cambiar de color
\item Freir las galletas hasta que empiecen a cambiar de color
\item Sofreir la cebolla, luego el ajo y luego el tomate. El sofrito debe de quedar bien cocido hasta que se forme una ``pasta''.
\item Moler la pimienta , oregano y clavo. Agregar al sofrito.
\item Freir los chiles ligeramente, solo por unos cuantos segundo por lado.
\item Licuar los chiles con caldo de pollo y piloncillo. Poner a hervir por \SI{20}{min}. 
\item Licuar todo junto. Agregar suficiente caldo de pollo para que quede aguadito.
\item Colar.
\item Agregar el pollo y \SI{1/2}{taza} de aceite.
\item Hervir por \SI{30}{min}, meneando ocacionalmente. Si está quedando muy espeso, agregar más caldo de pollo.
\end{enumerate}