\subsection{Tamales cuadrados tabasqueños}
\label{sec:tamales-cuadrados}

Receta de Lety, la tía de Kris.\\

\underline{Ingredientes}

\textbf{Relleno}
\begin{itemize}
\item 1 pavo
\item 5 ramas de epazote fresco
\item 5 chiles anchos
\item 5 chiles mulato
\item 4 chiles guajillo
\item \SI{1}{taza} de ajonjolí
\item \SI{1}{taza} de semilla de calabaza con cascarita
\item 1 cebolla morada chica
\item 1 cabeza de ajo chica
\item 2 tomates
\item \SI{186}{gr} de galleta soda (1 paquete)
\item 4 clavos de olor
\item \SI{1}{pizca} de orégano negro
\item 1 tortilla gruesa
\item \SI{2}{oz} de mole Doña María
\item \SI{\sim 1}{taza} de aceite
\item Sal
\item Agua
\end{itemize}

\textbf{Masa}
\begin{itemize}
\item \SI{6}{kg} de maiz
\item \SI{1}{l} de aceite o manteca de pavo
\item \SI{\sim 10}{l} de agua
\item Sal
\end{itemize}

\textbf{Armado}
\begin{itemize}
\item \SI{\sim 30}{mitades} de hija de platano (sin el tallo de en medio)
\end{itemize}

\underline{Instrucciones}

\textbf{Relleno}
\begin{enumerate}
\item Aliña el pavo cortándolo en \num{\sim 30} pedazos. Las alas se separan en dos y las piernas se parten en dos a lo ancho. 
\item Limpiar los chiles y cocer por unos minutos hasta que se aguaden.
\item En el aceite dorar los siguiente, en este orden y sin que se quemen:
\begin{enumerate}
\item Las galletas soda
\item La tortilla gruesa
\item La cebolla y los ajos (hasta que se acitrone la cebolla).
\item El epazote.
\item El ajonjolí y la semilla de calabaza. En este punto ya casi no debería de quedar aceite, agregar más de ser necesario.
\end{enumerate}
\item Picar el tomate y guisar en lo que quedó de aceite, el oregano y el clavo.
\item Moler el ajonjolí, los clavos y la semilla de calabaza.
\item Licuar todo con suficiente agua para que quede un mole aguadito.
\item Colar
\item Corregir de sal.
\item Dejar las piezas de pavo en el mole una noche.
\end{enumerate}

\textbf{Masa}
\begin{enumerate}
\item Cocer el maiz en suficiente agua y llevarlo a ebullición.
\item Apagar y colar el agua.
\item Moler. Dos veces si no queda fino a la primera.
\item Disolver con \SI{\sim 6}{l} de agua.
\item Colar con un colador fino, reservando el residuo.
\item Disolver con \SI{\sim 4}{l} de agua
\item Volver a colar. Recomendable primero colar con uno más grueso.
\item Agregar el aceite y la sal. Suficiente para que sepa saladita.
\item Poner a cocer a fuego medio-alto, meneando constantemenete y raspando el fondo. \SIrange{\sim 15}{20}{min} después notarás que empieza a ofrecer resistencia. Seguir meneando fuertemente por \SI{\sim 10}{min} hasta que de un hervor.
\item Dejar a que se enfríe por \SI{\sim 10}{min}.
\end{enumerate}

\textbf{Armado}
\begin{enumerate}
\item Suazar las hojas. Se pasar ligeramente por el fuego para que se suavicen. 
\item Cortar en trozos de \SI{\sim 40}{cm}x\SI{30}{cm}. Salen aproximadamente 3 por cada mitad de hoja, pero algunas se rompen.
\item Envoltura:
\begin{enumerate}
\item 1 taza de masa por tamal
\item Dos hojas por tamal
\item Guiso extra
\item Lado suave adentro
\end{enumerate}
\item Cocer en vaporera por \SI{3}{hr}. Deja enfriar antes de servir para que cuaje la masa.
\end{enumerate}