\subsection{Chocolate Mousse Cake}

Fuente: \href{https://www.lifeloveandsugar.com/2017/10/23/chocolate-mousse-cake/}{Life, love and sugar - Chocolate Mousse Cake}\\

\underline{Ingredientes}

\textbf{Pan}
\begin{itemize}
\item 230gr de harina
\item 420 gr de azúcar
\item 85 gr de cocoa
\item 2 $\sfrac{1}{4}$ cucharaditas de bicarbonato de sodio
\item $\sfrac{1}{2}$ cucharaditas de polvo para hornear
\item 1 cucharadita de sal
\item 1 taza de leche
\item $\sfrac{1}{2}$ tazas de aceite vegetal
\item 1 $\sfrac{1}{2}$ cucharaditas de vainilla
\item 2 huevos
\item 1 taza de agua caliente
\item Mantequilla para el molde
\end{itemize}

\textbf{Mousse}
\begin{itemize}
\item 4 yemas de huevo
\item $\sfrac{1}{4}$ tazas de azúcar
\item 1 $\sfrac{3}{4}$ tazas de heavy whipping cream
\item 225gr de chocolate en barra
\item 85 gr de azúcar glass 
\end{itemize}

\textbf{Betun}
\begin{itemize}
\item 1 $\sfrac{1}{4}$ tazas de heavy whipping cream
\item 35 gr de azúcar glass
\item 30 gr de cocoa
\item $\sfrac{1}{2}$ cucharadita de vainilla
\end{itemize}

\underline{Instrucciones}

\textbf{Pan}
\begin{enumerate}
\item Combinar ingredientes secos, luego líquidos, excepto el agua. Mezclar el agua al final.
\item Dividir en tres moldes redondos. Estos deben de tener matenquilla en el fondo y una capa de papel encerado.
\item Hornear a 350 F (180 C) por 20-25min, hasta que pase la prueba del palillo.
\end{enumerate}

\textbf{Mousse}
\begin{enumerate}
\item Mezclar las yemas, el azúcar y media taza de whipping cream. Cocinar a baño maría por \Sim 10-15min batiendo continuamente hasta que espese y adquiera volumen. El agua debe de mantenerse a \Sim 70C.
\item Derretir el chocolate a baño maría o en microondas.
\item Añadir a la primera mezcla y batir hasta que esté suave.  Dejar enfríar.
\item Batir el resto de la whipping cream (1$\sfrac{1}{4}$ de taza) con la azúcar glass hasta que forme picos.
\item Mezclar todo con movimiento envolventes, agregando de a poco a la vez.
\end{enumerate}

\textbf{Betún}
\begin{enumerate}
\item Bartir todo hasta que forme picos
\end{enumerate}

\textbf{General}
\begin{enumerate}
\item Hacer el pan y dejar enfriar.
\item Hacer el mousse.
\item Con la ayuda de un "cake collar" poner el mousse en medio de los panes. Dejar en el refrigerador por \Sim 5 horas.
\item Hacer el betún y poner una capa arriba.
\item Guardar en refrigerador.
\end{enumerate}