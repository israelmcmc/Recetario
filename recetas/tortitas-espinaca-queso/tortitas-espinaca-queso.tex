\subsection{Tortitas de espinaca con queso}

Basada en: \href{https://www.youtube.com/watch?v=r7K9CjKKr6k}{Jauja - Tortitas de Espinaca con Queso sin Capear}\\

\underline{Ingredientes}

\textbf{Caldo}
\begin{itemize}
\item \SI{300}{gr} de tomate (2 o 3)
\item 3 chiles serranos
\item 2 rebanadas de cebolla en plumas (\SI{\sim 80}{gr})
\item 1 diente de ajo picado
\item 1 pizca mejorana
\item 1 pizca tomillo
\item 1 hoja laurel
\item caldo de pollo (\num{\sim 1/2} cubito)
\item 1 pizca de sal
\item \SI{\sim 1}{cucharada} de aceite
\end{itemize}

\textbf{Tortitas}
\begin{itemize}
\item \SI{300}{gr} de espinaca
\item \SI{100}{gr} de queso rallado (e.g. manchego, chihuahua)
\item 1 pizca de sal
\item 1 pizca de pimienta
\item 1 huevo
\item 2 rebanadas de cebolla picada (\SI{\sim 80}{gr})
\item 1 cucharada de harina de trigo
\item 1 cucharadita de crema
\item \SI{\sim 1}{cucharada} de aceite
\end{itemize}

\underline{Instrucciones}

\textbf{Caldo}
\begin{enumerate}
\item Hervir los tomates y el chile hasta que hasta que revienten los jitomates
\item Licuar y colar.
\item Sofreir la cebolla.
\item Agregar lo licuado y el resto de los ingredientes.
\item Agregar un poco de agua en donde se cocieron los tomaes y hervir por unos minutos.
\end{enumerate}

\textbf{Tortitas}
\begin{enumerate}
\item Cocer la espinaca por \SI{\sim5}{min}.
\item Exprimir la espinaca
\item Sofreir la cebolla.
\item Mezclar todo.
\item Hacer 4 tortitas y dorarlas con poco aceite a fuego medio alto.
\item Servir con el caldo.
\end{enumerate}