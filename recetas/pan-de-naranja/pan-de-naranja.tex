\subsection{Pan de naranja}

Receta de Sra. Patricia Huerta, madre de la Sra. Dalia Ornelas\\

\underline{Ingredientes} 
\begin{itemize}
\item \SI{500}{gr} de mantequilla
\item \SI{500}{gr} de harina de trigo
\item 8 yemas  de huevo
\item 8 claras de huevo
\item \SI{350}{gr} de azúcar
\item 1 + \num{1/2} tazas de jugo de naranja
\item \SI{1/2}{cucharadita} de ralladura de naranja
\item \SI{3}{cucharaditas} de polvo de hornear\footnote{La receta original decía 2 cucharaditas. La mamá de Dalia cree que aunque funciona bien en la metropolis de Aguascalientes, para el nivel del mar es mejor usar \SI{1}{cucharadita} más. En alguno sitios recomiendan usar el doble para esas alturas.}
\item Mantequilla +  harina para el molde
\end{itemize}

\underline{Instrucciones}
\begin{enumerate}
\item Tritutar el azúcar en la licuadora en seco \footnote{Dalia creo que sólo es para hacer más fácil si mezcla a mano. Quizá se pueda usar azúcar glass directamente?}
\item Acremar la mantequilla
\item Agregar el azúcar y batir
\item Agregar las yemas una a una y batir
\item Mezclar la harina y el polvo de horner
\item Mezclar esto con el resto, poco a poco.
\item Batir las claras a punto de turrón
\item Incorporar las claras a la mezcla con movimiento envolventes.
\item Incorporar la ralladura de naranja y una taza de jugo.
\item Agregar más jugo de naranja hasta que masa se sienta fácil de batir.
\item Enharinar un molde y vaciar.
\item Hornear a 180C (360 F). hasta que pase la prueba del palillo (\SI{\sim 60}{min} para un molde de \SI{32}{cm} de diametro por \SI{8}{cm} de alto, según la mamá de Dalia, por lo menos en Aguas).
\end{enumerate}